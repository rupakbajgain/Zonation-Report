\chapter*{Abstract}
The study focuses on the soil bearing capacity and Liquefaction Potrntial of several locations in Kathmandu valley.
The allowable bearing capacities are calculated by using different parameters such as angle of internal friction, relative density, cohesion, which are determine through SPT-N values. Similarly for calculating liquefaction potential index unit weights, fineness content, etc. were used.
The study represents the examination of bearing capacity and LPI by using SPT-N values for areas of Kathmandu valley.
The soil BC conducted at depths 1.5 m, 3 m and 4.5m for square footing that is framed into geotechnical maps with the help of QGIS 3.x and for liquefaction the LPI was used for plotting. The BC was plotted based on shear and deflection strengths.
The significance shows the easy identification of soil BC of Kathmandu valley areas and helps to estimate appropriate designs for foundations and to determine the risk of liquefaction in certain areas. 
\\\\
\textbf{Keywords}: Bearing Capacity, PLAXIS 2D, Zonation, Borehole log, Liquefaction, LPI, GIS, Kathmandu
