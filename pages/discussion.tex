\chapter{Discussion}
Calculation of bearing capacity of soil has significant importance especially in urban areas like Kathmandu Valley where maximum land area is being used on building construction and the inadequacy of remaining land means high rise buildings are being constructed more and more. We can interpret the maps shown above to figure out the soil’s bearing capacity and liquefaction potential which determines the feasibility of construction in that area. 

The value of bearing capacity and LPI can be interpreted as shown below

\begin{itemize}
\item Region having bearing capacity $\le$ 100Kpa 			: Weak soil
\item Region having bearing capacity \textgreater 100Kpa and $\le$150Kpa	: Soft soil
\item Region having bearing capacity \textgreater 150Kpa and $\le$200Kpa	: Medium soil
\item Region having bearing capacity \textgreater 200Kpa and $\le$250Kpa	: Moderately hard soil
\item Region having bearing capacity \textgreater 250Kpa			: Hard soil
\end{itemize}

For LPI value of soil:
\begin{itemize}
\item LPI 0-2: Very low potential
\item LPI 2-5: Low potential
\item LPI 5-15: High potential
\item LPI >15: Very High potential
\end{itemize}

\section{Summary of bearing capacity at diffrent depths}
\subsection{Bearing Capacity at 1.5m depth}
At 1.5m depth, by shear failure criteria, the highest bearing capacity was found as 1135.052KPa at Kumaripati and the lowest bearing capacity was found at Bakhundol which was 75.880KPa. By deflection criteria, the highest value was at Satdobato as 216.65KPa and lowest value was 24.68 at Bakhundol.
For shear criteria, the soil at Bakhundol, Mulpani, Itachhe Tol, Jawalakhel, Khumaltar, Gwarko and Kupondole were found to be weak soils as their Bearing capacity value was less than 100KPa. The soil at Thapathali, Tangal, Harihar Bhavan, Lainchaur, Pumori and Kumaripati were found to be hard soil with bearing capacity greater than 250KPa.

\subsection{Bearing Capacity at 3m depth}
At 1.5m depth, by shear failure criteria, the highest bearing capacity was found as 1115.63KPa at Kumaripati and the lowest bearing capacity was found at Bakhundol which was 52.49KPa. By deflection criteria, the highest value was at Bhaisipati as 197.93KPa and lowest value was 31.88KPa at Bakhundol.
By shear failure criteria, Bakhundol, Mulpani, Itachhe Tol, Jawalakhel, Khumaltar, Gwarko and Kupondole were found to be weak soils as their Bearing capacity value was less than 100KPa. The soil at Thapathali, Balkumari, Bansbari, Tangal, Harihar Bhavan, Lainchour, Satdobato, Nagpokhari, Lainchaur, Kupondole, Pumori, Kumaripati, Thapathali were found to be hard soils with bearing capacity greater than 250KPa.

\subsection{Bearing Capacity at 4.5m depth}
At 4.5m depth, by shear failure criteria, the highest bearing capacity was found as 1700.13KPa at Kumaripati and the lowest bearing capacity was found at Bakhundol which was 54.51KPa. By deflection criteria, the highest value was at Bhaisipati as 181.51KPa and lowest value was 37.26KPa at Bakhundol.
By shear failure criteria, Anamnagar, Bakhundol, Itachhe Tol, Jawalakhel, Khumaltar, Gwarko and Kupondole were found to have weak soils as their Bearing capacity value was less than 100KPa. The soil at Thapathali, Balkumari, Mulpani, Tangal, Harihar Bhavan, Kuleswore, Lainchour, Nagpokhari, Mandikatar, Lainchaur, Kupondole, Pumori, Kumaripati, Thapathali and Kirtipur were found to be hard soils with bearing capacity greater than 250KPa.

\subsection{Liquefaction Potential}
The LPI value was found to be the highest at Tahachal with a calue of 50.30 and was the lowest at Kumaripati with a value of 0.00.
Bansbari, Bhaisepati, Gahanapokhari, Mulpani, Itachhe Tol, Jawalakhel, Satdobato, Nagpokhari, Mandikar and Kumaripati had very low liquefaction potential as their LPI value was found to be less than 2. Places like Thapathali, Anamnagar, Balkumari, Harihar Bhavan, Kuleswore,Tahachal, Khumaltar, Gwarko and kupandole had very high potential for liquefaction as their LPI value was greater than 15.