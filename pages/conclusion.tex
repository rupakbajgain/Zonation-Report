\chapter{Conclusion}
Using the 104 Borehole logs from 31 different locations throughout Kathmandu valley, mapping and zonation of bearing capacity and Liquefaction Potential Index were carried out. By analysis of those maps, following conclusions can be reached:

\begin{itemize}
\item The maps produced are powerful tools for the visualization of soil properties which can save time and cost to figure out the preliminary information of the soil required and the bearing capacity and LPI value to be expected.
\item The maps can serve as a guidance tools for engineers to figure out the suitability of any type of construction in the area, the need of treatments, type of foundation etc.
\item The bearing capacity of the soil depends upon the method used.
\item The LPI index map can be used as a hazard map on where liquefaction can be expected and where treatments need to be done.
\item Liquefaction potential was found to be strongly influenced by SPT value and the depth of water table.
\item The soil deposit of Kathmandu valley was found to be highly heterogeneous with extensively varying soil content and water table depth which means the bearing capacity and LPI value of the soil can change even in small distances.
\end{itemize}