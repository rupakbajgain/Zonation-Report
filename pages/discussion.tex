\chapter{Discussion}

Calculation of bearing capacity of soil has significant importance especially in urban areas like Kathmandu Valley where maximum land area is being used on building construction and the inadequacy of remaining land means high rise buildings are being constructed more and more. We can interpret the maps shown above to figure out the soil’s bearing capacity and liquefaction potential which determines the feasibility of construction in that area. 

The value of bearing capacity and LPI can be interpreted as shown below

\begin{itemize}
\item Region having bearing capacity $\le$ 100kPa : Weak soil
\item Region having bearing capacity \textgreater 100kPa and $\le$150kPa : Soft soil
\item Region having bearing capacity \textgreater 150kPa and $\le$200kPa : Medium soil
\item Region having bearing capacity \textgreater 200kPa and $\le$250kPa : Moderately hard soil
\item Region having bearing capacity \textgreater 250kPa : Hard soil
\end{itemize}

For LPI value of soil:
\begin{itemize}
\item LPI 0-2: Very low potential
\item LPI 2-5: Low potential
\item LPI 5-15: High potential
\item LPI \textgreater 15: Very High potential
\end{itemize}

\section{Summary of bearing capacity at diffrent depths}
\subsection{Bearing Capacity at 1.5m depth}
At 1.5m depth, by shear failure criteria, the highest bearing capacity was found as 1135.052kPa at Kumaripati and the lowest bearing capacity was found at Bakhundol which was 80.665kPa. By settlement strength criteria, the highest value was at Satdobato as 510.209kPa and lowest value was 32.327kPa at Thapathali. For settlement criteria, the soil at Bakhundol, Mulpani, Itachhe Tol, Harihar Bhawan, Kuleshwor, Tahachal, Nagpokhari, Gwarko and Kupondole were found to be weak soils as their Bearing capacity value was less than 100kPa. The soil at Balkumari, Bhaisepati, Jawalakhel, Satdobato and Kumaripati were found to be hard soil with bearing capacity greater than 250kPa.

\subsection{Bearing Capacity at 3m depth}
At 3m depth, by shear strength criteria, the highest bearing capacity was found as 1115.63kPa at Kumaripati and the lowest bearing capacity was found at Bakhundol which was 80.665kPa. By settlement criteria, the highest value was at Jawalakhel as 359.938 and lowest value was K38.642Pa at Thapathali. By shear failure criteria, Thapathali, Anamnagar, Bakhundol, Itachhe Tol, harihar Bhawan, Kuleshwor, Tahachal,Nagpokhari, Khumaltar, Gwarko, Kupondole and Kirtipur were found to be weak soils as their Bearing capacity value was less than 100kPa. The soil at Bhaisipati, Jawalakhel, Satdobato, Mandikatar, Kupondole and Kumaripati were found to be hard soils with bearing capacity greater than 250kPa.

\subsection{Bearing Capacity at 4.5m depth}
At 4.5m depth, by shear failure criteria, the highest bearing capacity was found as 1701.000kPa at Tangal and the lowest bearing capacity was found at Bakhundol which was 84.495kPa. By settlement criteria, the highest value was at Mandikatar as 281.543kPa and lowest value was 39.361kPa at Harihar Bhawan. By settlement criteria, Thapathali, Anamnagar, Bakhundol, Itachhe tol, Harihar Bhawan, Tahachal, Kupondole, Khumaltar, Gwarkho and Kirtipur were found to have weak soils as their Bearing capacity value was less than 100kPa. The soil at Mandikatar and Kumaripati were found to be hard soils with bearing capacity greater than 250kPa.

The net allowable bearing capacity was same as that of settlement method, except for 3 locations.

\subsection{Liquefaction Potential}
The LPI value was found to be the highest at Tahachal with a calue of 50.30 and was the lowest at Kumaripati with a value of 0.00. Bansbari, Bhaisepati, Gahanapokhari, Mulpani, Itachhe Tol, Jawalakhel, Lainchour, Satdobato, Nagpokhari and Kumaripati had very low liquefaction potential as their LPI value was found to be less than 2. Places like Thapathali, Anamnagar, Balkumari, Harihar Bhavan, Kuleswore, Tahachal, Khumaltar, Gwarko and Kupandole had very high potential for liquefaction as their LPI value was greater than 15.

\section{Similarity of LPI map with other susceptibility maps}
\subsection{UNDP/UNCHS}
\begin{itemize}
\item Most of our study area lies in area with high or very high potential.
\item Place like Tangal, Lainchour, Mandiktar lie in low susceptibility zone.
\end{itemize}

\subsection{JICA}
\begin{itemize}
\item Study area lies in high susceptible zone with high ground water level, ie. Alluvial Land, and Area formed by dry river bed so are mostly high and very high susceptible.
\end{itemize}

\subsection{Piya}
\begin{itemize}
\item Very low susceptibility at Itachhe tol. Bhaktapur.
\end{itemize}

\subsection{Subedi}
\begin{itemize}
\item Most study area has LPI \textgreater 1
\item Itache tol. Has LPI 0.479 (0.3-0.7).
\end{itemize}

\section{Relation of depth vs Bearing Capacity}
The Shear Bearing Capacity of soil increases with depth whereas the Deflection Bearing Capacity of soil remains about same with respect to depth. ie. force required for 25mm settlement.

\section{Area Summary}
For LPI most of area lies in High Potential Zone ie. 75%. For shear most area lies in Hard Soil Type and using settlement criteria most region lies in Soft and Medium Soil.

\section{Result from various methods}
Most shear methods gave similar results in mean. But the deviation was more in Plasix and least deviation was in Terzaghi.
Most deflection methods gave same result except Plasix which gave less capacity than other. The deviation in result was high in methods like Bowels and Teng and less in Plasix and Peck.

\section{Correlation between methods}
The shear methods like Terzaghi, Meyerhof, Hansen and Vesic have high +ve correlation. Plasix shear have no correlation with other results. Bowels, Teng , Meyerhof and Peck have higher correlation.  Plaxis settlement have weak correlation with other methods except with Plasix shear. So, these methods which have higher correlation gave similar results.