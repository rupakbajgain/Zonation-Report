\begin{thebibliography}{99}

\bibitem{dunn_fundamentals_1980}
Dunn, I. S., Anderson, L. R., \& Kiefer, F. W. (1980). \emph{Fundamentals of geotechnical analysis}. Jo Wiley \& Sons Incorporated.hn Wiley \& Sons Incorporated.

\bibitem{noauthor_nepal_1994}
Nepal Building Code (NBC-105), (1994). \emph{Seismic design of buildings in Nepal}. Kathmandu: Department of Building, Ministry of Physical Planning and Works, Government of Nepal.

\bibitem{arora_soil_2004}
Arora, K. R. (2021). \emph{SOIL MECHANICS AND FOUNDATION ENGINEERING}. STANDARD PUBLISHING.

\bibitem{danai_bearing_2018}
Danai, R. K., \& Acharya, I. P. (2018). \emph{Bearing Capacity Analysis and Zoning of Kathmandu for Shallow Foundations}. Journal of Advanced College of Engineering and Management, 4, 111-117.

\bibitem{mahoto_bearing_2012}
Mahato, S.K. (2012). \emph{Bearing Capacity zonation of urban and sub-urban area of Kathmandu and Lalitpur districts} (Master’s thesis, Pulchowk Campus, Institute of Engineering, Lalitpur, Nepal).

\bibitem{panthi_bearing_2018}
Jaya Ram Panthi. \emph{Bearing Capacity Mapping of Kathmandu Valley}. PhD thesis. Lalitpur, 2018.

\bibitem{arthur_characteristics_1936}
Casagrande Arthur. \emph{Characteristics of Cohesionless Soils Affecting the Stability of Slopes and Earthfills}. In: Journal of the Boston Society of Civil Engineers 23 (Jan. 1936), pp. 257–276.

\bibitem{hazen_study_1918}
Allen Hazen. \emph{A study of the slip in the Calaveras Dam}. In: (1918).

\bibitem{kulhawy_manual_1990}
F. H. Kulhawy and P. W. Mayne. \emph{Manual on Estimating Soil Properties for Foundation Design}. Itaca, New York: Cornell University, Aug. 1990.

\bibitem{r11}
Seed, H. B., \& Idriss, I. M. (1971). \emph{Simplified procedure for evaluating soil liquefaction potential}.

\bibitem{r12}
Journal of the Soil Mechanics and Foundations division, 97(9), 1249-1273.Seed, H. B., Idriss, I. M., \& Arango, I. (1983). \emph{Evaluation of liquefaction potential using field performance data. Journal of geotechnical engineering}, 109(3), 458-482.

\bibitem{r13}
Chen, Y. C. (1995). \emph{Effects of fines content on the relationship between maximum shear modulus and liquefaction}, NSC 84-2211-E-011-025.

\bibitem{r14}
Sherif, M. A., Tsuchiya, C., \& Ishibashi, I. (1977). \emph{Saturation effects on initial soil liquefaction}. Journal of the Geotechnical Engineering Division, 103(8), 914-917.

\bibitem{r15}
Seed, H. B., Mulilis, J. P., \& Chan, C. K. (1975). \emph{The effect of method of sample preparation on the cyclic stress-strain behavior of sands}. Rep. EERC, (75-18).

\bibitem{r16}
Peacock, W. H. and Seed, H. B., \emph{Sand liquefaction under cyclic loading simple shear conditions}, Journal of the Soil Mechanics and Foundations Division, ASCE; 94(SM3): 689-708, 1968.

\bibitem{r17}
Tokimatsu, K., \& Yoshimi, Y. (1983). \emph{Empirical correlation of soil liquefaction based on SPT N-value and fines content}. Soils and Foundations, 23(4), 56-74.

\bibitem{r18}
Gautam, D., Magistris, F.S., Fabbrocino, G., (2017) \emph{Soil Liquefaction in Kathmandu valley due to 25 April 2015 Gorkha, Nepal earthquake}. Soil Dynamics and Earthquake Engineering, 97:37-47

\bibitem{r19}
JICA, (2002). \emph{The study on earthquake disaster mitigation in the Kathmandu Valley} (Vol. 1)

\bibitem{r20}
JICA, (2018). \emph{The project for assessment of earthquake disaster risk for the Kathmandu valley in Nepal}. Final report, Volume I: Summary. 2-18-2-21.

\bibitem{r21}
UNDP/HMG/UNCHS (Habitat), (1994), \emph{Seismic hazard Mapping and Risk Assessment for Nepal}.

\bibitem{r22}
Wang, W.S. (1979). \emph{Some Findings in Soil liquefaction}. Water Conservancy and Hydroelectric Power Scientific Research Institute, Beijing, China.

\bibitem{r23}
Adhikari, Kalpana \& Subedi, Mandip \& Pokharel, Bigul \& Acharya, Indra \& Paneru, Harish \& Sharma, Keshab. (2019). \emph{Liquefaction of soil in Kathmandu Valley from the 2015 Gorkha, Nepal earthquake}.

\bibitem{r24}
Sakai, H., Fujii, R., \& Kuwahara, Y. (2002). \emph{Changes in the depositional system of the Paleo-Kathmandu Lake caused by uplift of the Nepal Lesser Himalayas}. Journal of Asian Earth Sciences, 20(3), 267-276.

\bibitem{r25}
Juang, C. H., \& Elton, D. J. (1991, August). \emph{Use of fuzzy sets for liquefaction susceptibility zonation}. In Proc. Fourth Intl. Conf. on Seismic Zonation (Vol. 2, pp. 629-636).

\bibitem{r26}
Iwasaki, T., Tokida, K. I., Tatsuoka, F., Watanabe, S., Yasuda, S., \& Sato, H. (1982). \emph{Microzonation for soil liquefaction potential using simplified methods}. In Proceedings of the 3rd international conference on microzonation, Seattle (Vol. 3, No. 2, pp. 1310-1330).

\bibitem{r27}
Youd, T. L., \& Idriss, I. M. (2001). \emph{Liquefaction resistance of soils: summary report from the 1996 NCEER and 1998 NCEER/NSF workshops on evaluation of liquefaction resistance of soils}. Journal of geotechnical and geoenvironmental engineering, 127(4), 297-313.

\bibitem{r28}
Subedi, M., Sharma, K., Upadhayay, B., Poudel, R. K., \& Khadka, P. (2013). \emph{Soil liquefaction potential in Kathmandu Valley. International Journal of Landslide and Environment}, 1(1), 91-92.

\bibitem{r29}
Subedi, M., Acharya, I. P., Sharma, K., \& Adhikari, K. (2015). \emph{Liquefaction of Soil in Kathmandu Valley From the 2015 Gorkha, Nepal, Earthquake}. Nepal Engineers’ Association. Tech J Special Issue Gorkha Earthquake, 108-115.

\bibitem{r31}
Bastola, A., \& Acharya, I. P. (2016). \emph{Liquefaction Susceptibility Mapping of Kathmandu Valley}. In Proceedings of IOE Graduate Conference (pp. 19-25).

\bibitem{idris_and_bolinger}
Idriss, I.M., \& Boulanger, R.W. (2008). \emph{Soil Liquefaction During Earthquakes}

\bibitem{idris_seed}
Idriss, I.M., \& Seed, H.B. (1968/7). \emph{Seismic Response of Horizontal Soil Layers}. Journal of Soil Mechanics and Foundation Division, 94(6), 1003-1031.

\bibitem{mey_book}
Das B.M. (2007). \emph{Theoretical foundation engineering}. "J. Ross publishing".

\end{thebibliography}