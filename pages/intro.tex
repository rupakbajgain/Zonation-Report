\chapter{Introduction}
\section{Bearing Capacity}
Due to rapid rate of urbanization in different parts of Kathmandu, a lot of structures from residential to commercial and structures that will have tourism value are being built which are also multi-storied. These structures however suitably conceptualized and designed from architectural, structural and usability viewpoints need to rest on Earth; transfer the entire load they are imposed to the ground. This act of transferring the load must be done so that the soil mass below the structure does not fail in shear and also does not settle differentially. The part of the structure, usually called sub-structure, as it is below the ground surface, is called foundation of the structure. Thus, foundation should be suitably designed bearing in mind the load it has to transfer to the ground and the type of soil it has to transfer load to. Here comes the need to determine bearing capacity of the soil upon which the foundation rests.

Bearing capacity of a soil is the load capacity of the superstructure that the designed foundation can dissipate on the ground without causing shear failure and excessive settlement of the soil. Many methods as developed by various scientists over the years can be used to determine the bearing capacity of the soil. These methods should be so chosen that the properties of the soil are rightly addressed. The bearing capacity thus determined can be used to design the foundation in similar type of scenario. Moreover, the value acquired by it can be used to judge roughly of how much strong the soil is to support a super-structure.

%\subsection{Background}
Various civil structures are being constructed everywhere. Stability of those depends upon different factors, the most important one of them being foundational stability. Foundational stability is mostly measured in terms of bearing capacity of soil. Various problems like liquefaction, periodic shrinkage and swelling, excessive settlement decreases the bearing capacity of soil. Due to this there appears to be difficulty in construction. So, we attempt to develop map that shows bearing capacity of Kathmandu valley.

Superstructures transfer force and moments through substructure to the underlying soil. So load should be within the safe limit of both foundation as well as soil. The safe limit of the foundation varies with the type of the foundation and material used. So it varies with the type of structure. But bearing capacity of the soil can be found out by using borehole log of that area.

In some other countries, bearing capacity zoning of highly urbanized cities has been done. But, in case of  Nepal,  no  such  proper  maps  have  been  prepared  yet.  Not  only  this,  but  also  in  the  developed countries,  people  are    using    modern software for modelling through  which  calculations  with  more realistic results have been obtained during their researches. So, in order to apply numerical modelling as more realistic and quick approaches in this modern world for determining soil strengths, this study have been attempted.

\section{Liquifaction}
%\subsection{Background}
Earthquake brings destruction of structures due to loss of stiffness of soil. Failure of dams, settlement of building, etc are brought along with destruction of soil. The phenomenon by which loss of strength occurs in soil is called soil liquefaction. It is generally associated with medium to fine grained saturated soil. During liquefaction material that is ordinarily a solid behaves like a liquid.

Casagrande\cite{arthur_characteristics_1936} made an attempt to explain liquefaction phenomenon in sandy soil. It is based on the concept of critical void ratio. It says that dense sand when subjected to shear tends to widen, loose sand under same condition tends to decrease. In between them there exists a point where volume doesn’t change this point is called critical void ratio. He explained there is excess foundation failure during earthquake because critical void ratio of sand decreases during vibrations of earthquake.

On the effective stress principles, liquefaction occurs when the pore water pressure  equals the total overburden stress. Mathematically,
	$\sigma'  =  \sigma_{vo} – u$
where,
	$\sigma'$=Effective stress
	$\sigma_{vo}$ =Total overburden stress
	$u$ =pore water pressure

Here, overburden stress remains constant but pore water pressure changes and may equal overburden pressure causing soil to have no bearing stress and acting as liquid I.e. liquefaction. Since critical void ratio is not constant but changes with confining pressure and volume change in dynamic loading conditions are different than one  directions static load conditions, critical void ratio concept may not be sufficient for quantitative evaluation of soil liquefaction.

Liquefaction is the phenomenon involving large settlements, sand boils, lateral spread, heavy-cracks or combination of all these occurring above or in the deposit of partially or fully saturated loose sand with just enough fines. For liquefaction to occur, it is necessary that the undrained conditions be established. The term “liquefying” was for the first time used by Allen Hazen\cite{hazen_study_1918}. Liquefaction only got serious attention in academics after 1964 Niigata and Alaska Earthquake. So, it’s a comparatively new topic and it is evolving in terms of scientific study. The susceptibility study of liquefaction in cities is a absolute necessity due to the larger risk it offers.
 
 \section{Scope}
Due to variation of bearing capacity of soil, soil test is essential for construction of civil engineering works. But, that is not possible for very small project or checking for large no of places. So, our map attempts to give tentative information about the bearing capacity of any place. This will help those projects to save funds and still create safe design. Due to rapid urbanization high rise buildings are of rapid demand in the Kathmandu valley as the space is very less. So for the preliminary study, our map can be used to find out the tentative area where the building can be constructed. This will help in saving time as well as money. This can be used as study materials by the upcoming researchers. 
 
The study of liquefaction susceptibility is ever evolving and the approach of the study, the factors that have direct influence in deciding whether liquefaction is a potential hazard within a locality or not are gradually being identified. This rate of rapid growth in the literature relating to liquefaction is bringing the susceptibility study very close to  practicable to identify places with appropriate level of hazard. In this light, as Kathmandu valley, is located in high seismic hazard zone, and the risk the hazards possess being absolutely high, it was relevant to conduct liquefaction susceptibility study in the valley. 
 
\section{Objectives}
Objectives of this study are:
\begin{itemize}
  \item	Finding bearing capacity using theoretical approach and Numerical Modeling in Plaxis-2D.
  \item	Create zoning of bearing capacity by plotting bearing capacity.
  \item	Getting to know the actual susceptibilty that liquefaction possess in various places of the valley.
  \item	Mapping the susceptibilty with respect to various locations for visualization.  
\end{itemize}

\section{Limitations}
\begin{itemize}
  \item	Water table is taken at surface for additional safety. This may not represent the exact nature of soil.
  \item	Study is limited for fixed footing, which in our case is 2*2 isolated footing.
  \item For PLASIX, axisymmetry is taken for model taking footing as circular.
  \item	Zoning map may not give accurate data because; it is merely interpolation of acquired data. It does not take into account every change in every point.
  \item The same SPT data was used to determine the results using various methods. Since each method has it's own set of parameters so results are diffrent in each case. Here the median was taken into account.
  \item	There were limited boreholes involved in the study, so interpolations between locations may not closely depict the scenario.
  \item	The study involves deterministic approaches which may not clearly be representative.
\end{itemize}